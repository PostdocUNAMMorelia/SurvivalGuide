\documentclass{article}

\usepackage[utf8]{inputenc}
\usepackage[T1]{fontenc}
\usepackage{lmodern}
\usepackage[margin=1.2in]{geometry}
\usepackage[bottom]{footmisc}
\usepackage{hyperref}

\title{Postdoc Survival Guide}
\author{}
\date{\today}

\begin{document}
\maketitle

\section{Issuing a student visa at the closest Mexican embassy}

Before arriving to Mexico and since you will be staying for more than 6 months you will need a student visa. Be really cautious and be clear that it is a student visa you want; if you have the wrong visa issued you can even face deportation. Before visiting the embassy, you will need to make an appointment with them. You will need to check how soon you can get an appointment with them as the visa will be valid for 6 months only (read final paragraph of the section). It does not matter which embassy you will visit for the issuance of your visa as long as you can prove your legal presence in the country. European citizens visiting other European countries do not need to prove that, however if a European citizen is visiting a Mexican embassy in the US, they will need to show their ESTA. After you have booked your appointment you will need to visit the embassy with the following documents (in general you show original or certified copies of documents accompanied by a photocopy of each):


\begin{itemize}
\item Valid passport and photocopy of the pages where the personal details and holder's photograph are found. Must be valid for at least an additional 6 months from the day it is presented.
\item Visa application (found at the embassy's website)
\item One passport size photograph (it must be in colour, with the face clearly visible without glasses, white background and be taken within the last month). 
\item Payment of consular fees (Visa) made in cash.
\item An original, dated and signed letter of acceptance from the school in Mexico that you intend to attend. This letter must be signed, on school's letter head and presented within 30 days from issuance. Electronic signatures, copies or scanned copies will not be accepted. You can find a template for the document in the section of \textbf{Document templates folder.}
\end{itemize}


\bigskip
\noindent
You will get your visa at most in 10 days. Most embassies issue them the same day. The visa will be valid for 6 months only. The entering date in Mexico will be the starting date for your residency card. The entering date \textbf{should} be before the starting date of your postdoc otherwise you will not be able to get paid for the first month! So if you are starting on the 1st of March, last date of entry can be the 28th of February. 

\bigskip
\noindent
Student visas are \textbf{single entry ONLY!} You have 30 days upon your arrival to exchange it for a Temporary Resident Student Card  at the nearest migration office (INM). The temporary resident student card will be valid for one year and multiple entries. Advice: visit the INM as soon as possible! Upon entering the country you will need to fill in the FMM migration form (landing card). You should keep this safe. Make sure that the immigration officer at the airport marks the FMM in the area that says ``canje'' or ``canjear.'' Check section \ref{Immi} for detailed paperwork you need to take with you at the INM.         


\section{Before arriving}

\subsection{Documents and copies}

\begin{itemize}
\item Passport: original and copy,
\item Visa: original and copy,
\item \ldots
\end{itemize}

\section{First visit to Immigration}\label{Immi}

\subsection{Things to do before the first visit}

By visiting the INM, you will exchange your student visa with a temporary resident card. The INM calls this process: Expedicion de documento migratorio por canje. You should visit the INM as soon as possible (first day if possible) as without the residency card: a) you cannot leave the country (in principle you can but for re-entering you will need to repeat the whole process of the visa issuance) b) you cannot get paid by the university and c) the university's insurance does not cover you. 


\subsection{Documents needed}

\begin{itemize}
\item The Forma Migratoria Multiple (FMM) which you completed before landing in Mexico and a copy of it. 
\item Pasport original and copy. 
\item Fill the Formato Basico and print it. The Formato Basico can be found \href{http://www.inm.gob.mx/complementos/FORMATO/Formato_Basico.pdf}{here.}
\item An online form with which you are asking the INM to change your student visa to a resident card. You can find the form \href{https://www.inm.gob.mx/tramites/publico/estancia.html}{here.} In the first steps of the form you need to select the following: 
 a) Que deseas hacer: canjear o reponer documento migratorio. b) Especifica: canje de FMM por tarjet de visitante o residente.  
\item Fill out and print the template \texttt{expedicion-de-documento-migratorio.tex} (found in the Document templates folder).
\item 3 photographs tipo infantil (size 2.5x3 cm.), two of the front and one of your right profile in white font without earrings, glasses and the ears clearly visible. There is a shop in the very centre where you can pay \$70 for two sets of these 3 photos (i.e. six in total). It is a good idea to have two sets just in case! The address is: Av Morelos Nte 150 and the shop is called Foto Magazin.
\item A letter of acceptance from your department that states your position and personal details. This can be the same as the one you presented to the embassy but with a different recipient. You can find a template of this in the \textbf{Document templates folder.}
\end{itemize}

For all above documents you will be asked to give your address. If you do not yet know where you will stay permanently, use whichever address you are currently staying at.  You should change it whenever you move.

As soon as you get the tarjeta you should directly ask for your CURP. For this you will need your passport, a photocopy and a photocopy of your tarjet. There is a Hiperlumen directly opposite the INM, so you can go make a copy and return to the INM for your CURP. The CURP is issued in 2 minutes and you should keep the paper they give you safe. 

\section{Useful document to keep copies of}

\end{document}
